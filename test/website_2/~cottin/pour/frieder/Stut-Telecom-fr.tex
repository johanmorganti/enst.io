\documentclass{article}
 \usepackage{a4} 
\usepackage{pstricks}
\usepackage{xspace}
\usepackage{pst-plot}
\usepackage{pst-node}
\usepackage{pst-coil}
\usepackage{subfigure}
\newcommand{\spin}{\textsc{spin}\xspace}
\newcommand{\sdl}{\textsc{sdl}\xspace}
\newcommand{\sdloo}{\textsc{sdl96}\xspace}
\newcommand{\promela}{\textsf{Promela}\xspace}
\newcommand{\cpp}{C\texttt{++}\xspace}
\newcommand{\cc}{\textsf{\footnotesize C}\xspace}
\newcommand{\java}{\textsf{Java}\xspace}
\newcommand{\Q}{{\cal Q}\xspace}
\newcommand{\F}{{\cal F}\xspace}
\newcommand{\A}{{\cal A}\xspace}

 
\begin{document}
%\title{Generation of certified code starting from specifications of
 % communicating systems}
\title{Generation of certified code  from specifications for
  communicating systems}


\author{
  {\footnotesize
    \begin{tabular}{ccc}
      Jean-Philippe~\textsc{Cottin}& & \\
      Ahmed ~\textsc{Serhrouchni}  
      &Siegfried~\textsc{L\"offler} & Omar~\textsc{Cherkaoui}\\
      ENST
                       %Ecole Nationale Sup�rieure des T�l�communications 
      & %Rechenzentrum der
      Universit\"at Stuttgart & Universit� du Qu�bec\\
                                %D�partement Informatique et R�seaux & Kommunikationssysteme/BelWue
                                %Entwicklung & D�partement Informatique\\
                                %46 rue Barrault, 75013 
      \textsc{Paris}  & % Allmandring 30, 70550
      \textsc{Stuttgart} 
      &
      \textsc{Montr�al}\\
      \{cottin,ahmed\}@enst.fr &
                                %siegfried.loeffler@rus.uni-stuttgart.de
      fl@lf.net
      & cherkaoui.omar@uqam.ca
    \end{tabular}
    }
  }
 
\date{}
\maketitle
 
\vskip 20mm

\abstract{
The complexity of software projects in the area of
  telecommunications systems demands for automated tool support. This
  can be provided if the underlying engineering methods have a formal
  foundation. Formal Description Techniques (FDTs) like
  \sdl\cite{SDL-ITU} have been successfully applied to the
  specifications of traditional communication protocols. Commercial
  and academic tools cover the whole process from specification
  towards implementation.

  We have contributed to this by proposing an extension of
  \spin\cite{spin} which can be used to generate implementations in C
  from \promela specifications.

  As a next step, we are examining whether object oriented techniques
  could be used to facilitate the process. For this, we examine two
  possibilities. 

  First, we want to study the use of an object oriented language for the
  implementation. Doing so we hope to improve automatically generated
  and validated code in a way so that it becomes more readable and
  maintainable.

  Second, we are examining whether \sdloo \cite{SDL96} as a formal
  language which is more common in industry could also be used as a
  source language for an automated generation of validated code.  \sdloo,
  compared to \promela, has the advantage that it is itself already
  object oriented and thereby allows to specify concepts like protocol
  services in a more reusable way.
}  
 
 
\section{Motivations}
The use of formal methods within the development of software allows to
make sure the design of a product is safe before actually implementing
it.

\begin{figure}[htb]
  \begin{center}
    \leavevmode
    \scaleboxto(7,8.1){
   \pspicture*(-2.5,3)(7,14)
    \rput(0,12.5){\rnode{Info1}
      {\psframebox[fillcolor=lightgray,fillstyle=solid]
        {\begin{tabular}{c} Informal \\ specification \\ of needs 
          \end{tabular}}}}
    \rput(0,10){\rnode{For}
      {\psframebox[fillcolor=lightgray,fillstyle=solid]
        {\begin{tabular}{c}Formal \\Specification 
          \end{tabular}}}}
    \rput(4,11.5)
    {\begin{tabular}{c} Until no errors \\ are found \end{tabular}}
   \rput(0,7){\ovalnode[fillcolor=lightgray,fillstyle=solid]{Quest}
%{\begin{tabular}{c} ~\\ ?\\ ~ \end{tabular}}}
     {\LARGE ?}} 
   \rput(4,10){\rnode{Val}
     {\psframebox[fillcolor=lightgray,fillstyle=solid]{Validation}}}
  %  \psframe[framearc=.3](-2,9)(6,11)
    \psframe[framearc=.3](-2,9)(6,13.4)
    \rput(0,5){\rnode{Log1}{
        \psframebox[framearc=.3,fillcolor=lightgray,fillstyle=solid]
        {\begin{tabular}{c} ~\\ Implementation\\ ~ \end{tabular}}}}
    
    \rput(0,9){\rnode{fin}{}}
 
    \ncline[arrowsize=4pt 5]{->}{Info1}{For}
    \ncline[arrowsize=4pt 5]{->}{For}{Val}
    \nccurve[arrowsize=4pt 5,angleA=90,angleB=0]{->}{Val}{Info1}
    \nccurve[arrowsize=4pt 5,angleA=90,angleB=45,
    linestyle=dotted,dotsep=2pt]{->}{Val}{For}

     \ncline[arrowsize=4pt 5]{->}{Quest}{Log1}
     \ncline[arrowsize=4pt 5]{->}{fin}{Quest}
      
     \psframe[framearc=.3](-2.5,3.25)(6.5,13.75)
\endpspicture
    }
    \caption{Software Development using Formal Description Techniques:
      After validation, how can one have a valid implementation of the
      problem?}
    \label{fig:question}
  \end{center}
\end{figure}


First, the model is validated (several methods can be used:
simulation, checking or even demonstration, cf fig\ref{fig:question}).
From this model one can then generate an implementation. 

Usually, this is done manually. However, there is a danger of
introducing errors and therefore this process should be done in the
most automatic possible manner. We want to contribute to this
by proposing methods and tools that help to automate this step.\\ 
 
We have started our study by working on a language of specification:
\promela\cite{spin}.  A specification described in \promela can be
simulated and checked by a public domain tool called
\spin developed and maintained per G Holzmann at \textit{Bell
  Laboratories}.  This language was primarily intended for
%created more specifically to 
validation of telecommunication protocols and distributed
systems.  We have chosen it because the source code of \spin
is free and its mechanism is quite easy to understand. The internal
representation of a system specified in \promela is very close to a
state machine.  Using this internal representation we were able to
extend this tool so that it allows to generate implementations as well
(cf. sec\ref{Promela}).\\ 


\section{What have we done so far ?}
\label{Promela}

\begin{figure}[htb]
  \begin{center}
    \leavevmode
    \scaleboxto(8,8){
      \pspicture*(-2.5,4)(7,14)
      \rput(0,12.5){\rnode{Info2}
        {\psframebox[fillcolor=lightgray,fillstyle=solid]{
            \begin{tabular}{c} Informal \\ specification \\ of needs 
            \end{tabular}}}}
      \rput(0,10){\rnode{For2}
        {\psframebox[fillcolor=lightgray,fillstyle=solid]
          {\begin{tabular}{c}Formal \\Specification \end{tabular}}}}
      \rput(4,11.5)
      {\begin{tabular}{c} Until no errors \\ are found \end{tabular}}
      \rput(0,8){\rnode{ST}
        {\psframebox[fillcolor=lightgray,fillstyle=solid]
          {\begin{tabular}{c}State Machine  \end{tabular}}}} 

      \rput(4,6){\rnode{Log2}
        {\psframebox[framearc=.3,fillcolor=lightgray,fillstyle=solid]
          {\begin{tabular}{c} ~\\ Implementation\\ ~ \end{tabular}}}}
      \rput(0,6){\rnode{Val2}
        {\psframebox[framearc=.3,fillcolor=lightgray,fillstyle=solid]
          {\begin{tabular}{c} ~\\ Validation\\ ~ \end{tabular}}}}
      
      \rput(0,9){\rnode{fin}{}}
      
      \ncline[arrowsize=4pt 5]{->}{Info2}{For2}
      \nccurve[arrowsize=4pt 5,angleA=0,angleB=0]{->}{For2}{Info2}
      \ncline[arrowsize=4pt 5]{->}{For2}{ST}
      \Aput{automatic (\spin)}
      
      \ncline[arrowsize=4pt 5]{->}{ST}{Val2}
      \nccurve[arrowsize=4pt 5,angleA=0,angleB=90]{->}{ST}{Log2}
      \Aput{our extension}
      
      \psframe[framearc=.3](-2.5,4.25)(6.5,13.75)
      \endpspicture
      }
    \caption{The model we have implemented}
    \label{fig:proposition}
  \end{center}
\end{figure}

In Holzmann's \promela/\spin the same tool is used for the simulation
and the creation of the validator.  We developed an extension of this
tool to create an implementation in \texttt{C} language. We managed to
keep close to the validated code by reusing the internal data
structures to build the implementation. More information about this
extension can be found in \cite{ams}.

Our extension also allows to generate distributed code. For this, we
use \textsc{af\_unix} domain sockets for the inter--process
communication between \textsc{unix} processes.\\ 

This approach enabled us for example to create a validated implementation
for a well known problem, the \texttt{"Steam Boiler Control"}
\cite{spin97}.\\ 

However, we found that our approach still is problematic. On the one
hand, the \promela language is quite academic and not used very often
for industrial projects. On the other hand, the generated \cc code is
neither readable nor maintainable any more. Because of this, we study
the possibility to use an object-oriented language.


\section{How Formal Methods can be brought closer to current 
software development techniques?}

Having seen that the most important disadvantage of our current tool
is that of the unreadability of the generated code, object oriented
programming techniques came to our mind as a possible solution.
Thinking about object oriented languages, we see two fields where 
they could be applied in the software development process. First
we thought of simply modifying or \promela to \texttt{C} compiler so that
it generates object oriented code. Then it came to our mind that
this might be even more interesting if we could also use object
oriented techniques for the specification. 

% We might have to modify this a bit:
For the automatically generated code, we think that the core
technique will still be communicating state machines, since they
are the heart of the validation techniques. Therefore
we feel it to be necessary to use them as a starting point
for our reflections, for the object oriented generated code
as well as for the object oriented specification.

\subsection{Using \textit{OO} language for implementations}
\label{Java}

Looking at object oriented programming languages for the generated
implementations, \java has the advantage that it already supports
"threads" and therefore seems to be a very suitable for implementing
protocols.\\ 

For the representation in \java, we need to define \textit{objects for
 state machines}. The (\textsl{Finite States Machines}) are part of
the automata theory. A finite state machine has always a finite number
of states, one of which is the initial state. Such an automaton may be
induced to change its state by receiving one of the signals defined
for it in its \textit{input alphabet}. A \promela \texttt{proctype} is
a communicating extended finite-state machine i.e. "extended" by
variables and give the possibility to exchange signals. These machines
are a \textit{sextuple}:
$$CESFM=(\Q, q_0, A,f,Z,X)$$

 where:

\begin{itemize}
\item $\Q$ is a set of states,
\item $q_0$ the initial state,
\item $A$ the \textit{input alphabet} (the \textit{input} signals)
\item $Z$ is a set of variables with $\F(Z)$  the set of logic
  formulas on the variables of $Z$ (the results of conditions ...) and
   $\A(Z)$ the set of actions on  the variables of $Z$ (calculus,
   \texttt{printf}...)
\item $X$ is the \textit{output alphabet}  (the \textit{output}
  signals)
\item and $f$ is the set of transitions defines in  $\Q \times A \times
\F(Z) \times \A(Z) \times X^* \times \Q$.
\end{itemize} 

For each of the operations in the sense of \promela, we
intend to use a \java class stimulated by variables and channels
definitions (see fig\ref{fig:java}).\\ 

\begin{figure}[htb]
  \begin{center}
    \leavevmode
    \scaleboxto(10,6){
      \pspicture*(-2.5,4.5)(11,13)

      \pspolygon(0,10)(1,12)(2,10)
      \rput(1,11){VAR} 
      \rput(1,10.25){\footnotesize channels}

      \pszigzag[coilwidth=.5,coilarm=.2,linearc=.1]{<->}(1.5,11)(5,11)

      \rput(6,12){\ovalnode{If1}{~}}
      \rput(8,12.5)
      {\footnotesize \texttt{Class If (cond, state\_machine.. )}}
      \psline{->}(8,12.5)(7.5,11.5)
      \rput(6,11){\rnode{If}
        {\psframebox[fillcolor=lightgray,fillstyle=solid]{IF}}}
      \rput(6,10){\ovalnode{If3}{~}}
      \rput(8,10){\ovalnode{If4}{~}}
      \rput(8,9){\rnode{If-fin}{$\cdots$}}

      \ncline{If1}{If}
      \nccurve[arrowsize=4pt 5,angleA=270,angleB=90]{If}{If3}
      \nccurve[arrowsize=4pt 5,angleA=270,angleB=90]{If}{If4}
      \psframe[framearc=.3,linestyle=dashed,dotsep=1pt](5,9.5)(9,11.5)
      \ncline{If4}{If-fin}

      \pspolygon(.5,5)(1.5,7)(2.5,5)
      \rput(1.5,6){VAR} 
      \rput(1.5,5.25){\footnotesize channels}
      \pszigzag[coilwidth=.5,coilarm=.5,linearc=.1]{<->}(2,6)(5,7)

      \rput(6,8){\rnode{Do}
        {\psframebox[fillcolor=lightgray,fillstyle=solid]{DO}}}
      \rput(6,7){\ovalnode{Do3}{~}}
      \rput(6,6){\ovalnode{Do4}{~}}
      \rput(6,5){\rnode{Do-fin}{$\cdots$}}

      \ncline{If3}{Do}
      \ncline{Do}{Do3}
      \ncline{Do3}{Do4}
      \nccurve[linearc=.9,angleA=300,angleB=50]{->}{Do4}{Do}
      \psframe[framearc=.3,linestyle=dashed,dotsep=1pt](5,5.3)(7.5,8.5)
      \ncline{Do4}{Do-fin}
      \endpspicture
      }
    \caption{How the State Machine can be projected to Objects}
    \label{fig:java}
  \end{center}
\end{figure}



\subsection{Specifying the model in an object oriented formal language}

\label{jpapi}
Looking on possibilities to use object oriented techniques to write
the specifications, we envisage different ways to do that.  One idea
is to no longer use a formal language at all but to provide a special
set of \java classes for the description of protocols. These classes
(lets call them JPAPI\footnote{for \java Protocol API} for example)
would primarily serve protocol implementers to write their
implementations. At the same time, we could then write a compiler that
allows a translation of this subset of \java to some kind of state
machine representation which can be validated. During this process, we
would reduce the detail level of the \java specification to the bare
minimum we need, i.e. we would for example reduce the messages
exchanged to their meaning for the protocol machine and remove things
like the exchanged data.

\label{SDL}
Further, we consider an \textit{OO} formal language as a starting
point for the software development process: \sdloo\cite{SDL96}.

This new dialect of \sdl contains language concepts covering the basic
aspects of object orientation (identity, classification, polymorphism,
inheritance,\dots).

Basically, this languages has two characteristics:
\begin{itemize}
\item From a \texttt{static} point of view (cf fig\ref{blocks}):
  we have a hierarchical concept (\texttt{block}, \texttt{process}
  that  can be translated in \textit{OO} (with inheritance, 
 copy method,...)
\item From a \texttt{dynamic} point of view  (cf fig\ref{process}):
 the specification of the \sdl-processes concerns the dynamic behavior
 of a system. It is based on the concept of extended finite-state
 machines like shown in section \ref{Java}.
\end{itemize}

Some commercial tools (SDT, ObjectGeode, ...) can already generate the
code of a system after the validation of a specification. However
those don't provide an \textit{OO} implementation whereas the initial
\sdloo specification was.  A translation from \sdl to \java (or \cpp)
in which the structure of a system can be found in terms of class
would be useful.

The processes could be described with a class in which there is a method
\texttt{"transition"} which contains the behavior of the
state-machine.

\begin{figure}[htb]
  \begin{center}
    {\mbox
      {\subfigure[Hierarchy of blocks\label{blocks}]
        {\scaleboxto(7,5){
            \pspicture*(-.5,-.5)(8,6)
            
            \psframe(-.5,-.5)(7.8,5.5)
            \psframe(0,0)(3,2)
            \rput(1.5,1.75){Block1}
                                %   \pspolygon(.5,.25)(.25,.5)(.25,1)(.5,
                                %   1.25)(1,1.25)(1.25,1)(1.25,.5)(1,.25)  
            \pspolygon(.4,.2)(.2,.4)(.2,.8)(.4,1)
            (1.2,1)(1.4,.8)(1.4,.4)(1.2,.2)  
            \rput(.8,.6){\footnotesize Process1}

            \pspolygon(1.8,.6)(1.6,.8)(1.6,1.2)(1.8,1.4)
            (2.6,1.4)(2.8,1.2)(2.8,.8)(2.6,.6)
            \rput(2.2,1){\footnotesize Process2}
            

            \psline[arrowsize=1pt 1.5]{->}(1.4,.4)(2,.4)(2,.6)
            \psline(3,1)(5.5,1)
            \psline{-o}(2.8,1)(3,1)    
            
            \psline{->}(3,1)(5,1)
            \psline(5.5,1)(5.5,3)
            \psline{<-}(5.5,1.5)(5.5,2)
            \rput(6.3,1.7){Channel1}
            \psline(1.5,2)(1.5,4)
            \psline(1.5,4)(4,4)
            \psline{->}(1.5,4)(2.5,4) \rput(2.5,4.5){Channel2}
            \psframe(4,3)(7,5)
            \rput(5.5,4){Block2}
            \psline{->}(7,4)(7.4,4)
            \psline{-o}(7.4,4)(7.8,4)
            \endpspicture
            }
          }%\quad
        \subfigure[State machine representing a process\label{process}]
        {  
          {\scaleboxto(5,2.7){
              \pspicture*(-2.3,-7.3)(15.3,2.3)
              \psframe(-2.2,-7.2)(15.2,2.2)
      \rput(13,2){process Process1}
       \pspolygon(12,1)(13.8,1)(13.8,.8)(14,.8)
       (13.8,1)(14,.8)(14,0)(12,0)
       \rput(12.5,.5){\footnotesize dcl ...}

      \psframe[framearc=1](-2,1)(0,2)
      \psline{->}(-1,1)(-1,.6)
      \rput(-1,.1){\footnotesize Idle}
      \psframe[framearc=1](-2,-.4)(0,.6) %%
      \psframe[framearc=1](1,1)(3,2)
      \rput(2,1.5){\footnotesize Idle}
      \psline(2,1)(2,.4)
      \psline(2,.6)(6.8,.6)
      \psline(4.4,.6)(4.4,.4)
      \psline(6.8,.6)(6.8,.4)
      \pspolygon(1,.4)(3,.4)(2.5,-.1)(3,-.6)(1,-.6)
      \rput(2,-.1){\footnotesize CR}
      \psline(2,-.6)(2,-1)
      \psframe(1,-1)(3,-2)
      \rput(2,-1.5){\footnotesize Sdu!id:=CR}
      \psline(2,-2)(2,-3)
      \pscircle(1.4,-2.5){.4}
      \rput(1.4,-2.5){\footnotesize grs0}
      \psline{->}(1.8,-2.5)(2,-2.5)
      \pspolygon(1,-4)(1,-3)(2.5,-3)(3,-3.5)(2.5,-4)
      \rput(2,-3.33){\tiny MDATreq}
      \rput(2,-3.66){\tiny (Sdu)}
      \psline{->}(2,-4)(2,-4.4)
      \psframe[framearc=1](1,-5.4)(3,-4.4)
      \rput(2,-4.9){\footnotesize Idle} %
      \pspolygon(3.4,.4)(5.4,.4)(4.9,-.1)(5.4,-.6)(3.4,-.6)
      \rput(4.4,.07){\tiny DT}
      \rput(4.4,-.27){\tiny (Num,d)}
      \psline(4.4,-.6)(4.4,-1)
      \psframe(3.4,-2)(5.4,-1)
      \rput(4.4,-1.25){\tiny Sdu!id:=DT,}
      \rput(4.4,-1.5){\tiny Sdu!Num:=Num,}
      \rput(4.4,-1.75){\tiny Sdu!Data:=d}
      \psline{->}(4.4,-2)(4.4,-2.4)
      \pscircle(4.4,-2.8){.4}
      \rput(4.4,-2.8){\footnotesize grs0}  %
      \pspolygon(5.8,.4)(7.8,.4)(7.3,-.1)(7.8,-.6)(5.8,-.6)
      \rput(6.8,.07){\tiny MDATind}
      \rput(6.8,-.27){\tiny (Sdu)}
      \psline(6.8,-.6)(6.8,-1)
      \pspolygon(6.8,-1)(7.8,-1.5)(6.8,-2)(5.8,-1.5)
      \rput(6.8,-1.5){\tiny Sdu!id}
      \psline(6.8,-2)(6.8,-2.4)
      \psline(6.8,-2.2)(14,-2.2)
      \psline(9.2,-2.2)(9.2,-2.4)
      \psline(11.6,-2.2)(11.6,-2.4)
      \psline(14,-2.2)(14,-2.4)
      \rput(6.8,-2.9){\footnotesize CC}
      \rput(9.2,-2.9){\footnotesize AK}
      \rput(11.6,-2.9){\footnotesize DR}
      \rput(14,-2.9){\footnotesize ELSE}
      \psline(6.8,-3.4)(6.8,-3.8)
      \psline(9.2,-3.4)(9.2,-3.8)
      \psline(11.6,-3.4)(11.6,-3.8)
      \psline{->}(14,-3.4)(14,-3.8)
      \pspolygon(5.8,-3.8)(7.3,-3.8)(7.8,-4.3)(7.3,-4.8)(5.8,-4.8)
      \rput(6.8,-4.3){\footnotesize CC}
      \psline(6.8,-4.8)(6.8,-5.8)
      \pscircle(6.2,-5.3){.4}
      \rput(6.2,-5.3){\footnotesize grs1}
      \psline{->}(6.6,-5.3)(6.8,-5.3)
      \psframe[framearc=1](5.8,-6.8)(7.8,-5.8)
      \rput(6.8,-6.3){\footnotesize Idle} %
      \pspolygon(8.2,-3.8)(9.7,-3.8)(10.2,-4.3)(9.7,-4.8)(8.2,-4.8)
      \rput(9.2,-4.3){\tiny AK(Sdu!Num)}
      \psline{->}(9.2,-4.8)(9.2,-5.2)
      \pscircle(9.2,-5.6){.4}
      \rput(9.2,-5.6){\footnotesize grs1} %
      \pspolygon(10.6,-3.8)(12.1,-3.8)(12.6,-4.3)(12.1,-4.8)(10.6,-4.8)
      \rput(11.6,-4.3){\footnotesize DR}
      \psline{->}(11.6,-4.8)(11.6,-5.2)
      \pscircle(11.6,-5.6){.4}
      \rput(11.6,-5.6){\footnotesize grs1} %
      \psframe[framearc=1](13,-4.8)(15,-3.8)
      \rput(14,-4.3){\footnotesize Idle} %
      \endpspicture

              }
            }
          }
        }
}
      \caption{Modularization of a system in SDL/GR - Refining the
        model from blocks down to processes}
      \label{fig-sdl}
    \end{center}
  \end{figure}


\section{Outlook and Conclusions} % and Conclusions ?

The most important result we have seen in our studies is that in fact
all kind of formal validation processes can somehow be reduced to some
kind of communicating state machines.

The next step we feel to be necessary to tackle the problem that the
validated state machine code usually is not readable and maintainable
for software engineers will be the introduction of a well structured
definition of a core set of state machine classes in an object
oriented programming language.

Once we will have established those classes, we plan to develop
further tools that are based on them. The first thing that would be
good to have at that stage would be a compiler that allows a
translation of the state machines coded in \java or \cpp back to
\promela.  Backtranslating the implementation of the state machines to
\promela would increase the degree of trust one has in the generated
code by allowing to validate it once again in its final stage.

Another idea we want to develop further is the development of classes
that allow a simple and fast description of protocols in general. By
basing those "protocol development classes" on the state machine
classes, a whole framework for protocol development could be
established step--by--step.



\begin{thebibliography}{xx}
\bibitem{spin}
G.~Holzmann.
\newblock {\it Basic \spin manual}.
\newblock All the sources and documentation about \spin
are available on http://netlib.bell-labs.com/netlib/spin/.

\bibitem{SDL-ITU}
ITU-T~Recommendation Z.100:.
\newblock {\it Specification and description language \sdl}.
\newblock ITU General Secretariat, Sales  Section, Places des Nations,
CH-1211 Geneva 20.


\bibitem{spin97}
S.~L\"offler, A.~Serhrouchni,
\newblock Creating a Validated Implementation of the Steam Boiler Control"
\newblock Proceedings of the Third \spin Workshop \textit{(SPIN97)}
\newblock Twente University, Enschede, The Netherlands, 4/1997.

\bibitem{SDL96}
J.~Ellsberger, D.~Hogrefe, A.~Sarma,
\newblock {\it SDL: Formal Object-oriented Language for
 Communicating systems}
\newblock Prentice Hall, 1997
 
\bibitem{Stutt96}
S.~L\"offler, A.~Serhrouchni,
\newblock Protocol Design: From Specification to Implementation
\newblock \textit{5th Open Workshop on High Speed Networks}
\newblock Co-organized by University of Stuttgart and ENST. March 96.

\bibitem{ams}
S.~L\"offler, A.~Serhrouchni,
\newblock Creating implementations from \promela models,
\newblock in \texttt{The \spin Verification System}, 
\newblock DIMACS: Series in Discrete Mathematics and Theoretical
Computer Science, Volume \textbf{32},
\newblock American Mathematical Society 
\end{thebibliography}



\end{document}

%%% Local Variables: 
%%% mode: latex
%%% TeX-master: t
%%% End: 
